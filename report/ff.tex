\chapter{FFのデータベース}
\section{概要}
このWebサイトでは,FinalFantasyシリーズのタイトル一覧を表示する.
機能として,ナンバリング一覧表示,派生作品一覧表示,詳細表示,データの編集,追加,削除がある.
ファイルはff.js,ff\_db1.ejs,ff\_db2.ejs,ff\_detail.ejs,ff\_edit.ejs,ff\_add.htmlの5つのファイルで構成されている.
\section{データ構造}
ff.jsには,ファイナルファンタジーの作品情報を保存する配列(fainalfantasy)がある.
データ構造の使用例は,表\ref{ff_data}に示す.nameは,タイトル名を示す.idは,作品を一意に指定できるように番号が付与される.
seriesは,派生元の作品を示す.yearは,作品の発売年月を示す.mainは,第一作目なのかを示す.
deviceは,作品の対応ハードを示す.explanationでは,作品の説明を示す.
\begin{table}[H]
    \centering
    \caption{ffのデータ構造の使用例}
    \begin{tabular}{|c|c|c|c|c|c|c|}
        \hline
        name & id & series & year & main & device & explanation \\ \hline
        Final Fantasy & 1 & FinalFantasyI & 1987-12 & true & ファミリーコンピューター & FF シリーズ初タイトル \\ \hline
        Final Fantasy II & 2 & FinalFantasyII & 1988-12 & true & ファミリーコンピューター & 熟練度システム \\ \hline
        Final Fantasy III & 3 & FinalFantasyIII & 1990-04 & true & ファミリーコンピューター & 初のジョブシステム \\ \hline
        Final Fantasy XIII & 4 & FinalFantasyXIII & 2009-12 & true & PS3 & オプティマシステム \\ \hline
        Final Fantasy XII-2 & 5 & FinalFantasyXIII & 2011-12 & false & PS3 & FF13 の続編 \\ \hline
        \begin{tabular}[c]{@{}l@{}}FINALFANTASYTACTICS \\ THE IVALICE CHRONICLES\end{tabular} & 6 & FinalFantasyothers & 2025-09 & false & PS5 & タクティカル RPG の金字塔 \\ \hline
    \end{tabular}
\end{table}

